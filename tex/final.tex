\documentclass{beamer}
\usetheme{Boadilla} 
\setbeamercovered{invisible}
\setbeamertemplate{navigation symbols}{} 
%\useoutertheme{infolines} 


\usepackage{graphicx}

\setbeamertemplate{frametitle continuation}{} 
\usepackage{subfigure}
\usepackage{caption}
\usepackage{array}
\usepackage{bm}
\usepackage{epsfig}
\usepackage[utf8]{inputenc}
\usepackage{amsmath}
\usepackage{xcolor,colortbl}

\usepackage{multicol}
\usepackage{wasysym}


\DeclareMathOperator{\median}{median}
\DeclareMathOperator{\cor}{cor}
\newcommand{\norm}[1]{\left|\left|#1\right|\right|}

% deal with spaces in absolute paths
\usepackage[space]{grffile}
\graphicspath{{C:/Users/Yered/Dropbox/Harvard/Winter 2014/CdeC/Slides/Introduction/figures/}}

\usepackage[scaled]{helvet}
\usepackage[round]{natbib}

\usepackage{tikz}
\def\checkmark{\tikz\fill[scale=0.4](0,.35) -- (.25,0) -- (1,.7) -- (.25,.15) -- cycle;} 

\def\mathrlap{\mathpalette\mathrlapinternal} 
\def\mathclap{\mathpalette\mathclapinternal}
\def\mathllapinternal#1#2{\llap{$\mathsurround=0pt#1{#2}$}}
\def\mathrlapinternal#1#2{\rlap{$\mathsurround=0pt#1{#2}$}}
%\usetikzlibrary{fit}
%\tikzset{%
%  highlight/.style={rectangle,fill=red!15,
%    fill opacity=0.45,inner sep=0pt}
%}
%\newcommand{\tikzmark}[2]{\tikz[overlay,remember picture,
%  baseline=(#1.base)] \node (#1) {#2};}
%
%\newcommand{\Highlight}[1][submatrix]{%
%    \tikz[overlay,remember picture]{
%    \node[highlight,fit=(left.north west) (right.south east)] (#1) {};}
%}


\begin{document}
\title[Final]{Explorando el Transcriptoma con Datos de Expresi\'{o}n Gen\'{e}tica\\
\vspace{0.5cm}
Final}
\author{Yered Pita-Ju\'{a}rez}
\institute[CdeC M\'{e}rida]{}
\date{9/1/2015}


\begin{frame}
\titlepage
\end{frame}

\begin{frame}{Cursos en Línea}
\begin{itemize}
\item 7.00x: Introduction to Biology 7.00x (MITx)\\
\url{https://courses.edx.org/courses/MITx/7.00x/2013_SOND}
\item Coursera: R Programming\\
\url{https://www.coursera.org/course/rprog}
\item PH525x: Data Analysis for Genomics (HarvardX)\\
\url{https://courses.edx.org/courses/HarvardX/PH525x/1T2014/}
\end{itemize}
\end{frame}
\nocite{*}

\begin{frame}[allowframebreaks]
\frametitle{Libros}
\small
\bibliographystyle{authordate2}
\bibliography{CdeC_refs}
\end{frame}

\begin{frame}{Contacto}
Yered Pita-Juárez\\
\texttt{ypitajuarez@fas.harvard.edu}\\
\texttt{yered.h@gmail.com}\\
\vspace{1cm}
Ingrid Rodriguez Gutierrez\\
\texttt{ingridrdz20@gmail.com}
\end{frame}

\end{document} 
