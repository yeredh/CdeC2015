\documentclass{beamer}
\usetheme{Boadilla} 
\setbeamercovered{invisible}
\setbeamertemplate{navigation symbols}{} 
%\useoutertheme{infolines} 

\usepackage[utf8]{inputenc}
\usepackage{graphicx}

\setbeamertemplate{frametitle continuation}{} 
\usepackage{subfigure}
\usepackage{caption}
\usepackage{bm}
\usepackage{epsfig}

\usepackage{amsmath}
\usepackage{xcolor,colortbl}

\usepackage{multicol}
\usepackage{wasysym}

\usepackage{hyperref}
\usepackage{float}

\usepackage{array}
\newcolumntype{L}[1]{>{\raggedright\let\newline\\\arraybackslash\hspace{0pt}}m{#1}}
\newcolumntype{C}[1]{>{\centering\let\newline\\\arraybackslash\hspace{0pt}}m{#1}}
\newcolumntype{R}[1]{>{\raggedleft\let\newline\\\arraybackslash\hspace{0pt}}m{#1}}

\usepackage[T1]{fontenc}
\usepackage{tikz}
\usetikzlibrary{shadows}

\newcommand*\keystroke[1]{%
  \tikz[baseline=(key.base)]
    \node[%
      draw,
      fill=white,
      drop shadow={shadow xshift=0.25ex,shadow yshift=-0.25ex,fill=black,opacity=0.75},
      rectangle,
      rounded corners=2pt,
      inner sep=1pt,
      line width=0.5pt,
      font=\scriptsize\sffamily
    ](key) {#1\strut}
  ;
}

% deal with spaces in absolute paths

\usepackage[space]{grffile}
\graphicspath{{C:/Users/Yered/Dropbox/Harvard/Winter 2014/CdeC/Slides/Introduction/figures/}}

\usepackage[scaled]{helvet}
\usepackage[round]{natbib}


\begin{document}
%\title[Explorando el Transcriptoma]{Explorando el Transcriptoma con Datos de Expresi\'{o}n Gen\'{e}tica}
%\author{Yered Pita-Ju\'{a}rez}
%\institute[CdeC M\'{e}rida]{}
%\date{??/??/2015}
\title[Introducción a R \& RStudio]{Explorando el Transcriptoma con Datos de Expresi\'{o}n Gen\'{e}tica\\
\vspace{0.5cm}
Introducción a R \& RStudio}
\author{Yered Pita-Ju\'{a}rez}
\institute[CdeC M\'{e}rida]{}
\date{6/1/2015}


\begin{frame}
\titlepage
\end{frame}

\begin{frame}[fragile]{Caracteres}
\begin{itemize}
\item Define una variable de caracteres usando apostrofes
\item De otra manera R busca una variable definida con ese nombre
\begin{verbatim}
> m = "Manzanas"
> m
[1] "Manzanas"
> n = peras
Error: object 'peras' not found
\end{verbatim}
\end{itemize}
\end{frame}

\begin{frame}[fragile]{Caracteres}
\begin{itemize}
\item No podemos hacer cálculos con variables de caracteres
\begin{verbatim}
> m+2
Error in m + 2 : non-numeric argument to binary operator
\end{verbatim}
\item Pero tenemos operaciones para manipular caracteres
\begin{verbatim}
> toupper(m)
[1] "MANZANAS"
> tolower(m)
[1] "manzanas"
\end{verbatim}
\end{itemize}
\end{frame}


\begin{frame}[fragile]{Vectores}
\begin{itemize}
\item Se puede crear un vector con la función \verb=c()=
\begin{verbatim}
> vec1 = c(1,4,6,8,10)
> vec1
[1] 1 4 6 8 10
\end{verbatim}
\item Los elementos de un vector se pueden acceder y modificar por su índice \verb=[i]=
\begin{verbatim}
> vec1[5]
[1] 10
> vec1[3] = 12
> vec1
[1] 1 4 12 8 10
\end{verbatim}
\end{itemize}
\end{frame}

\begin{frame}[fragile]{Vectores}
\begin{itemize}
\item El comando \verb=seq= es una forma de crear un vector a partir de una secuencia
\begin{verbatim}
> vec2 = seq(from=0, to=1, by=0.25)
> vec2
[1] 0.00 0.25 0.50 0.75 1.00
\end{verbatim}
\end{itemize}
\end{frame}

\begin{frame}[fragile]{Matrices}
\begin{itemize}
\item Son vectores con 2 dimensiones
\begin{verbatim}
mat=matrix(data=c(9,2,3,4,5,6),ncol=3)
> mat
[,1] [,2] [,3]
[1,] 9 3 5
[2,] 2 4 6
\end{verbatim}
\item \verb=data= contiene las entradas de la matrix, \verb=ncol= especifica el número de columnas y \verb=nrow= especifica el número de hileras
\item Los elementos de una matriz se acceden por su posición \texttt{[fila,columna]}
\begin{verbatim}
> mat[1,2]
[1] 3
\end{verbatim}
%\item Para acceder a toda una hilera
%\begin{verbatim}
%> mat[2,]
%[1] 2 4 6
%\end{verbatim}
%\item Para acceder a toda una columna
%\begin{verbatim}
%> mat[,1]
%[1] 9 2
%\end{verbatim}
\end{itemize}
\end{frame}

\begin{frame}[fragile]{Matrices}
\begin{itemize}
\item Para acceder a toda una hilera
\begin{verbatim}
> mat[2,]
[1] 2 4 6
\end{verbatim}
\item Para acceder a toda una columna
\begin{verbatim}
> mat[,1]
[1] 9 2
\end{verbatim}
\end{itemize}
\end{frame}

\begin{frame}[fragile]{Data Frames}
\begin{itemize}
\item Una matriz con columnas etiquetadas
\begin{verbatim}
> t = data.frame(x = c(11,12,14),y = c(19,20,21),
			    z = c(10,9,7))
> t
   x  y  z
1 11 19 10
2 12 20  9
3 14 21  7
\end{verbatim}
\item Las columnas se pueden acceder por su nombre
\begin{verbatim}
> mean(t$z)
[1] 8.666667
> mean(t[["z"]])
[1] 8.666667
\end{verbatim}
\end{itemize}
\end{frame}


\begin{frame}[fragile]{Data Frames}
\begin{block}{Ejercicio}
\begin{itemize}
\item Crea un script para crear 3 vectores (\texttt{x1,x2,x3}) con 100 números aleatorios usando \verb=rnorm=
\item Crea un \texttt{data frame} llamado \verb=t= con 3 columnas (\texttt{a,b,c}) que contengan respectivamente \texttt{x1,x1+x2,x1+x2+x3}
\item Usa las siguientes functiones \texttt{plot(t)} y \texttt{summary(t)}
\item ¿Cómo interpretarías los resultados?
\end{itemize}
\end{block}
\begin{figure}[H]
\centering
\includegraphics[scale=0.11]{laptop.jpeg}
\end{figure}
\end{frame}

%\begin[fragile]{frame}{Data Frames}
%\begin{block}{Ejercicio}
%\begin{itemize}
%\item Crea un script para crear 3 vectores (\texttt{x1,x2,x3}) con 100 números aleatorios usando \verb=rnorm=
%\item Crea un \texttt{data frame} llamado \verb=t= con 3 columnas (\texttt{a,b,c}) que contengan respectivamente %\texttt{x1,x1\+x2,x1\+x2\+x3})
%\item Usa las siguientes functiones \texttt{plot(t)} y \texttt{summary(t)}
%\item ¿Cómo interpretarías los resultados?
%\end{itemize}
%\end{block}
%\begin{figure}[H]
%\centering
%\includegraphics[scale=0.15]{laptop.jpeg}
%\end{figure}
%\end{frame}

\begin{frame}[fragile]{Listas}
\begin{itemize}
\item Una coleccción de objectos
\item Si es lista de vectores, los vectores no tienen que ser del mismo tamaño
\item Crear una lista 
\begin{verbatim}
> L = list(one=1, two=c(1,2),five=seq(0, 1, length=5))
> L
$one
[1] 1
$two
[1] 1 2
$five
[1] 0.00 0.25 0.50 0.75 1.00
\end{verbatim}
\end{itemize}
\end{frame}



\begin{frame}[fragile]{Listas}
\begin{itemize}
\item Nombres de los componentes de la lista
\begin{verbatim}
names(L)
[1] "one" "two" "five"
\end{verbatim} 
\item Operaciones con los elementos de la lista
\begin{verbatim}
> L$five + 10
[1] 10.00 10.25 10.50 10.75 11.00
\end{verbatim}
\end{itemize}
\end{frame}

\begin{frame}[fragile]{Listas}
\begin{itemize}
\item Las listas pueden contener differentes tipos de variables
\begin{verbatim}
Lst <- list(nombre="Jose", esposa="Maria",
            no.hijos=3,edad.hijos=c(4,7,9))
\end{verbatim}
\item Podemos acceder a los componentes de la lista usando el nombre del componente o el indice
\begin{columns}
    \begin{column}{0.45\textwidth}
\begin{verbatim}
> Lst$nombre
[1] "Jose"
> Lst$esposa
[1] "Maria"
> Lst$edad.hijos
[1] 4 7 9
\end{verbatim}
    \end{column}
    \begin{column}{0.45\textwidth}
\begin{verbatim}
> Lst[[1]]
[1] "Jose"
> Lst[[2]]
[1] "Maria"
> Lst[[4]]
[1] 4 7 9
\end{verbatim}
    \end{column}
\end{columns}
\end{itemize}
\end{frame}

\begin{frame}[fragile]{Listas}
\begin{itemize}
\item Si uno de los componentes tiene varios elementos, podemos accederlos usando el indice correspondiente
\begin{verbatim}
> Lst$edad.hijos[1]
[1] 4
> Lst[[4]][1]
[1] 4
\end{verbatim}
\end{itemize}
\end{frame}

\begin{frame}[fragile]{Listas}
\begin{block}{Ejercicio}
Crea una lista con tu siguientes datos
\begin{itemize}
\item Nombre
\item Apellido
\item Edad
\item Escuela
\end{itemize}
\end{block}
\begin{figure}[H]
\centering
\includegraphics[scale=0.1]{laptop.jpeg}
\end{figure}
\end{frame}

\begin{frame}[fragile]{NA}
\begin{itemize}
\item Trabajando con datos reales, a veces tenemos observaciones que no están disponibles
\item R representa estos datos faltantes con el valor \verb=NA=
\begin{verbatim}
> obs=c(1,2,NA)
\end{verbatim}
\item En general, no es posible calcular estadisticas de datos incompletos
\begin{verbatim}
> mean(obs)
[1] NA
\end{verbatim}
\item Sin embargo, podemos optar por ignorar los datos que nos hacen falta
\begin{verbatim}
> mean(obs,na.rm=TRUE)
[1] 1.5
\end{verbatim}
\end{itemize}
\end{frame}



\begin{frame}[fragile]{Combinando Dos Objectos: match, merge}
\begin{itemize}
\item Tenemos dos \texttt{data frames} con datos acerca de unas ratas: cada uno con un ID diferente para cada rata
\end{itemize}
\footnotesize
\begin{verbatim}
ratas <- data.frame(id = paste0("rat",1:10),  
                    sexo = factor(rep(c("F","M"),each=5)),
                    peso = c(2,4,1,11,18,12,7,12,19,20),
                    longitud = c(100,105,115,130,95,150,165,180,190,175))
> ratas
      id sexo peso longitud
1   rat1    F    2      100
2   rat2    F    4      105
3   rat3    F    1      115
4   rat4    F   11      130
5   rat5    F   18       95
6   rat6    M   12      150
7   rat7    M    7      165
8   rat8    M   12      180
9   rat9    M   19      190
10 rat10    M   20      175
\end{verbatim}
\end{frame}


\begin{frame}[fragile]{Combinando Dos Objectos: match, merge}
\begin{itemize}
\item Tenemos dos \texttt{data frames} con datos acerca de unas ratas: cada uno con un ID diferente para cada rata
\end{itemize}
\footnotesize
\begin{verbatim}
ratasTabla <- data.frame(id = paste0("rat",c(6,9,7,3,5,1,10,4,8,2)),
                         IDsecreto = 1:10)
> ratasTabla
      id IDsecreto
1   rat6         1
2   rat9         2
3   rat7         3
4   rat3         4
5   rat5         5
6   rat1         6
7  rat10         7
8   rat4         8
9   rat8         9
10  rat2        10
\end{verbatim}
\end{frame}

\begin{frame}[fragile]{Combinando Dos Objectos: match, merge}
\begin{itemize}
\item Queremos ordenar el \texttt{data frame} \verb=ratas= usando el \texttt{IDsecreto} de \verb=ratasTabla=.
\item \verb=match= te da por cada elemento del primer vector, el indice que corresponde al segundo vector
\begin{verbatim}
> match(ratasTabla$id, ratas$id)
 [1]  6  9  7  3  5  1 10  4  8  2
\end{verbatim}
\item El \texttt{data frame} que quieres reordenar es el segundo argumento de \verb=match=
\end{itemize}
\end{frame}

\begin{frame}[fragile]{Combinando Dos Objectos: match, merge}
\begin{itemize}
\item El \texttt{data frame} que quieres reordenar es el segundo argumento de \verb=match=
\footnotesize
\begin{verbatim}
> ratas[match(ratasTabla$id, ratas$id),] 
      id sexo peso longitud
6   rat6    M   12      150
9   rat9    M   19      190
7   rat7    M    7      165
3   rat3    F    1      115
5   rat5    F   18       95
1   rat1    F    2      100
10 rat10    M   20      175
4   rat4    F   11      130
8   rat8    M   12      180
2   rat2    F    4      105
\end{verbatim}
\end{itemize}
\end{frame}

\begin{frame}[fragile]{Combinando Dos Objectos: match, merge}
\begin{itemize}
\item El \texttt{data frame} que quieres reordenar es el segundo argumento de \verb=match=
\footnotesize
\begin{verbatim}
> cbind(ratas[match(ratasTabla$id, ratas$id),],ratasTabla)
      id sexo peso longitud    id IDsecreto
6   rat6    M   12      150  rat6         1
9   rat9    M   19      190  rat9         2
7   rat7    M    7      165  rat7         3
3   rat3    F    1      115  rat3         4
5   rat5    F   18       95  rat5         5
1   rat1    F    2      100  rat1         6
10 rat10    M   20      175 rat10         7
4   rat4    F   11      130  rat4         8
8   rat8    M   12      180  rat8         9
2   rat2    F    4      105  rat2        10
\end{verbatim}
\end{itemize}
\end{frame}

\begin{frame}[fragile]{Combinando Dos Objectos: match, merge}
\begin{itemize}
\item Tambien podemos combinar los \texttt{data frames} usando la funcion \verb=merge=
\footnotesize
\begin{verbatim}
> ratasMerge <- merge(ratas, ratasTabla, by.x="id", by.y="id")
> ratasMerge[order(ratasMerge$IDsecreto),]
      id sexo peso longitud IDsecreto
7   rat6    M   12      150         1
10  rat9    M   19      190         2
8   rat7    M    7      165         3
4   rat3    F    1      115         4
6   rat5    F   18       95         5
1   rat1    F    2      100         6
2  rat10    M   20      175         7
5   rat4    F   11      130         8
9   rat8    M   12      180         9
3   rat2    F    4      105        10
\end{verbatim}
\end{itemize}
\end{frame}

\begin{frame}[fragile]{Instalando Paquetes}
\begin{figure}[H]
\centering
\includegraphics[scale=0.25]{R-logo.png}
\end{figure}
\textbf{Comprehensive R Archive Network (CRAN)}
\begin{itemize}
\item Principal repositorio de paquetes para R
\item 5,800 paquetes y 120,000 funciones (Junio 2014)
\end{itemize}
\begin{figure}[H]
\centering
\includegraphics[scale=0.35]{BioClogo.png}
\end{figure}
\textbf{Bioconductor}
\begin{itemize}
\item Repositorio de paquetes para R
\item Análisis de datos genómicos
\end{itemize}
\end{frame}

\begin{frame}{Instalando Paquetes}
\begin{figure}[H]
\centering
\includegraphics[scale=0.24]{GitHub-Logo.jpg}
\end{figure}
\textbf{GitHub}
\begin{itemize}
\item Repositorio popular para muchos proyectos open source
\item Algunos paquetes de R
\end{itemize}
\end{frame}


\begin{frame}[fragile]{Instalando Paquetes de GitHub}
\begin{figure}[H]
\centering
\includegraphics[scale=0.24]{GitHub-Logo.jpg}
\end{figure}
\begin{itemize}
\item Necesitamos instalar primero \texttt{devtools}
\begin{verbatim}
install.packages("devtools")
\end{verbatim}
\item Vamos a instalar \texttt{dagdata} de GitHub
\begin{verbatim}
library(devtools)
install_github("genomicsclass/dagdata")
\end{verbatim}
\end{itemize}
\end{frame}

\begin{frame}[fragile]{Instalando Paquetes de Bioconductor}
\begin{figure}[H]
\centering
\includegraphics[scale=0.35]{BioClogo.png}
\end{figure}
\begin{itemize}
\item Para instalar el paquete base de Bioconductor
\begin{verbatim}
source("http://bioconductor.org/biocLite.R")
biocLite()
\end{verbatim}
\item Para instalar un paquete de Bioconductor
\begin{verbatim}
source("http://bioconductor.org/biocLite.R")
biocLite("NombreDelPaquete")
\end{verbatim}
\item Más información para instalar y actualizar paquetes de Bioconductor: 
\url{http://bioconductor.org/install/}
\end{itemize}
\end{frame}

\begin{frame}[fragile]{Instalando Paquetes}
\begin{block}{Ejercicio}
Instala los siguientes paquetes de Bioconductor
\texttt{
\begin{itemize}
\item affy
\item oligo
\item GEOquery
\item rae230a.db
\item AnnotationDbi
\item biomaRt
\end{itemize}}
\end{block}
\begin{figure}[H]
\centering
\includegraphics[scale=0.1]{laptop.jpeg}
\end{figure}
\end{frame}

\begin{frame}[fragile]{Instalando Paquetes}
\begin{block}{Ejercicio}
Carga el paquete \verb=affy=, y accede a su vignette.
\end{block}
\begin{figure}[H]
\centering
\includegraphics[scale=0.2]{laptop.jpeg}
\end{figure}
\end{frame}


\begin{frame}[fragile]{Vignettes}
\begin{itemize}
\item Documentos de ayuda para paquetes de R
\item Requeridos para cada paquete de Bioconductor
\item Manuales con ejemplos
\item ¿Cuáles vignettes están disponibles?
\begin{verbatim}
vignette(package="Biobase")
\end{verbatim}
\item Accediendo al pdf
\begin{verbatim}
vignette("ExpressionSetIntroduction")
\end{verbatim}
\item Navegando las vignettes
\begin{verbatim}
browseVignettes(package="Biobase")
\end{verbatim}
\end{itemize}
\end{frame}



%\begin{frame}[fragile]{}
%\end{frame}

%
%
%\begin{frame}[fragile]{}
%\begin{block}{Ejercicio}
%
%\end{block}
%\begin{figure}[H]
%\centering
%\includegraphics[scale=0.15]{laptop.jpeg}
%\end{figure}
%\end{frame}
%
%




\end{document} 

