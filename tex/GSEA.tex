\documentclass{beamer}
\usetheme{Boadilla} 
\setbeamercovered{invisible}
\setbeamertemplate{navigation symbols}{} 
%\useoutertheme{infolines} 

\usepackage[utf8]{inputenc}
\usepackage{graphicx}

\setbeamertemplate{frametitle continuation}{} 
\usepackage{subfigure}
\usepackage{caption}
\usepackage{bm}
\usepackage{epsfig}

\usepackage{amsmath}
\usepackage{xcolor,colortbl}

\usepackage{multicol}
\usepackage{wasysym}

\usepackage{hyperref}
\usepackage{float}

\usepackage{array}
\newcolumntype{L}[1]{>{\raggedright\let\newline\\\arraybackslash\hspace{0pt}}m{#1}}
\newcolumntype{C}[1]{>{\centering\let\newline\\\arraybackslash\hspace{0pt}}m{#1}}
\newcolumntype{R}[1]{>{\raggedleft\let\newline\\\arraybackslash\hspace{0pt}}m{#1}}

\usepackage[T1]{fontenc}
\usepackage{tikz}
\usetikzlibrary{shadows}

\newcommand*\keystroke[1]{%
  \tikz[baseline=(key.base)]
    \node[%
      draw,
      fill=white,
      drop shadow={shadow xshift=0.25ex,shadow yshift=-0.25ex,fill=black,opacity=0.75},
      rectangle,
      rounded corners=2pt,
      inner sep=1pt,
      line width=0.5pt,
      font=\scriptsize\sffamily
    ](key) {#1\strut}
  ;
}

% deal with spaces in absolute paths

\usepackage[space]{grffile}
\graphicspath{{C:/Users/Yered/Dropbox/Harvard/Winter 2014/CdeC/Slides/Introduction/figures/}}

\usepackage[scaled]{helvet}
\usepackage[round]{natbib}


\begin{document}
\title[GSEA]{Explorando el Transcriptoma con Datos de Expresi\'{o}n Gen\'{e}tica\\
\vspace{0.5cm}
GSEA}
\author{Yered Pita-Ju\'{a}rez}
\institute[CdeC M\'{e}rida]{}
\date{8/1/2015}


\begin{frame}
\titlepage
\end{frame}

\begin{frame}[fragile]{Gene Set Enrichment Analysis}
\begin{itemize}
\item Prueba de expresion diferencial usando grupos de genes
\item La idea principal es usar grupos de genes pre definidos
\item Definir los grupos de genes para ayudar a interpretar los resultados
\item Analizar muestras del ALL
\item leucemia linfoide aguda en celulas B
\item Dos grupos: con on sin la mutación BCR/ABL
\end{itemize}
\footnotesize
\begin{verbatim}
library("ALL")
data("ALL")
bcell = grep("^B", as.character(ALL$BT))
moltyp = which(as.character(ALL$mol.biol) %in% c("NEG", "BCR/ABL"))
ALL_bcrneg = ALL[, intersect(bcell, moltyp)]
ALL_bcrneg$mol.biol = factor(ALL_bcrneg$mol.biol)
\end{verbatim}
\end{frame}



\begin{frame}[fragile]{Gene Set Enrichment Analysis}
\begin{itemize}
\item Filtrado: seleccionar los genes que estan por arriba del percentil del 50\% de la variabilidad
\begin{verbatim}
library("hgu95av2.db")
library("genefilter")
ALLfilt_bcrneg = nsFilter(ALL_bcrneg, var.cutoff=0.5)$eset
\end{verbatim}
\item KEGG: base de datos que contiene genes en rutas biologicas
\item Definir los grupos de genes basados en rutas biologicas
\begin{verbatim}
library("GSEABase")
gsc = GeneSetCollection(ALLfilt_bcrneg,
                        setType=KEGGCollection())
Am = incidence(gsc)
dim(Am)
nsF = ALLfilt_bcrneg[colnames(Am),]
\end{verbatim}
\end{itemize}
\end{frame}


\begin{frame}[fragile]{Gene Set Enrichment Analysis}
\begin{itemize}
\item Solo vamos a considerar las rutas biologicas con mas de 10 genes
\begin{verbatim}
selectedRows = (rowSums(Am)>10)
Am2 = Am[selectedRows, ]
\end{verbatim}
\item Vamos a usar la prueba de GSEA
\begin{verbatim}
library("Category")
set.seed(123)
NPERM = 1000
pvals = gseattperm(nsF, nsF$mol.biol, Am2, NPERM)
pvalCut = 0.025
lowC = names(which(pvals[, 1]<=pvalCut))
highC = names(which(pvals[, 2]<=pvalCut))
\end{verbatim}
\end{itemize}
\end{frame}

\begin{frame}[fragile]{Gene Set Enrichment Analysis}
\begin{itemize}
\item Determinar el nombre de las rutas biologicas que tienen expression diferencial
\begin{verbatim}
library("KEGG.db")
head(getPathNames(lowC))
head(getPathNames(highC))
\end{verbatim}
\end{itemize}
\end{frame}

\begin{frame}[fragile]
\begin{block}{Ejercicio}
Trata de interpretar las rutas biologicas que estan diferencialmente expresadas.
\begin{verbatim}
"Homologous recombination"
"Ribosome"
"Complement and coagulation cascades"
"Lysosome"
"Axon guidance"
"Pathogenic Escherichia coli infection"
"Shigellosis"
"Viral myocarditis"
\end{verbatim}
\end{block}
\end{frame}


\end{document}

\begin{frame}[fragile]{Gene Set Enrichment Analysis}
\begin{itemize}
\end{itemize}
\end{frame}


\begin{frame}[fragile]{Comparaciones Múltiples}
\end{frame}

